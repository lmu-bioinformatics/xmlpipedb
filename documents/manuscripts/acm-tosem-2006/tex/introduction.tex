\section{Introduction}
We started out with a goal, to build a tool that will take an XML file from a gene information storage giant like Uniprot and convert it to a \genmapp~gene database. At first this appeared to be trivial to solve by using existing tools like \hyperjaxb2~and \hibernate. However, in the begining stages of putting together our first version we found many problems that are common in XML to relational database conversion. Our tool was to be as generic as possible to be able load and convert gene information from a number of different sources, but because the nuances we discovered were source specific we could not build an all encompassing tool. We broke up the task into several modules in order to facilitate both the generic conversion and source specific customizations that were needed.  Each module would take part in the task of converting gene information from one form to another.

\xmlpipedb~is a suite of tools for managing, querying, importing and exporting information from XML data, to a relational database, and finally to a \genmapp~gene database. This paper describes the motivations behind our original goal.  We will step you through the process we experienced and show you the solution that has reached our goal. We will look at each piece that makes this tool chain a success.  We will explore how our solution solves both the data transformation required in the bioinfomatics community, as well as its applicability in the more general case within computing and computer science communities.  Finally, we will analyze and critique our own work and explore further goals and future directions for \xmlpipedb.
