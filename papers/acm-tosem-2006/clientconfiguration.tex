\subsection{Client Configuration}
In order to make the tools as easy to use as possible the issue of Hibernate properties configuration had to be addressed. Hibernate supports a wide variety of databases and data access methods. While advantageous for technical people, this can make configuration a daunting task for others. We choose an approach that would be modular, allowing the configuration component to be plugged into any application seamlessly.  

The challenge was to display all the properties in a convenient and straight forward way with a maximum of flexibility, to accomodate any future changes. The project was broken down into tasks: redefining the hibernate properties approach; creating data objects to model the properties; creating an engine to read and write properties; and finally creating a GUI to allow users to easily manipulate the properties. This led logically to breaking the component in to four distinct objects: HibernateProperty, HibernatePropertiesModel, ConfigurationEngine and ConfigurationPanel. 

A thorough examination of the hibernate properties showed that properties could be easily broken into categories and types. Categories are platforms, general and connection pools. Platforms contain the different database platforms supported by hibernate via direct connection, e.g. PostgreSQL, MySQL, etc. Connection pools are the different connection pools, such as C3PO or Apache DBCP, that are supported. General are properties that apply to any setup, regardless of whether it is a direct connection or a connection pool. Types are used to identify the specific type of property within the category, e.g. PostgreSQL is a type within the platforms category and C3PO is a type within the connection pools category.

The data structures to support this breakdown consist of a HibernateProperties object to store each individual property and a HibernatePropertiesModel as a container for the HibernateProperty objects. HibernateProperty objects consist of a category, type, name and value, as well as an indicator of whether the property is saved or not. This allows both saved and unsaved properties to be stored in the same model. The ConfigurationEngine loads the default data structure for properties along with default values into a HibernatePropertiesModel. The saved properties are then read in and layered on top of the default data model, replacing the default values for those properties that were saved. The resulting model contains all the possible properties, and the values for those that were saved. 

The model is used by the ConfigurationPanel to render the a GUI for the user to view and/or edit the properties. However, if a different GUI were desired, the  ConfigurationEngine, HibernatePropertiesModel and HibernateProperty objects could be reused. When properties are saved, a model object is passed back to the controller, which then stores the properties in a file.



