\section{Overall Design and Approach}
To understand the overall design of XMLPipeDB we must first state our initial goal:  To create a reusable tool set that given genomic sequencing data for an organism in xml and a schema for that xml document could out put a working GenMAPP gene database for that organism.  Given the many different source organizations, and consequently different schemas,  for this xml data we wanted to be able to automate this process as much as possible.  In the initial research phase of the project we managed to find that quite a fair amount of work had already been done by the open source community to reach our final goal.  The JAXB reference implementation contained an xml schema compiler to automatically create JAXB objects given an xml schema.   Hyberjaxb could then be used to annotate these objects and automatically generate Hibernate mappings for them.  We could then use hibernate and the generated hibernate mappings to place these objects into a relational  database.  Finally we could export the needed fields from our relational database to create our GenMAPP gene database. 

Initially the project seemed simple and we attempted to design it as a single unit, but after some experimentation we noted that some additional post processing was necessary and it made more since to break the project up into three different stages.  The first stage which we called xsd2db, described more thoroughly in \ref{xsd2db}, would be a general purpose JAXB object and hibernate mapping generator that would take an xml schema file as input.  