\subsubsection{Ad Hoc Queries}

\paragraph{Introduction}
A major factor in any research tool is data analysis. To this end, we are providing users of xmlpipedbutils with two query tools: an SQL analyzer and an HQL analyzer. Each of these views allows for a different view of the data and allows each user to utilize the query language they are most familiar with.

\paragraph{Design and Implementation}

\subparagraph{GUI Design}

The query panel is split into three sections; each separated by a splitter. On the left of the user interface is the query section. The radio and command buttons are on the right. The right side consists of two radio buttons and two buttons. The two buttons let the user choose between standard query language (SQL) and hibernate query language (HQL) queries. 

The left side is further split into two other sections. The north section contains a text area in which the query can be typed. The bottom of the panel contains two components: a table and a tree. How each view is populated depends on whether we are executing an SQL query versus an HQL query.

The command buttons are labeled ``Clear'' and ``Execute Query''. Clicking on clear will erase the text from the query text are and clear any data from the results table and tree. Clicking the execute button will execute the query in the text area and populate either the tree or the table with the returned data. We will discuss how the data is displayed further in the following sections.

\subparagraph{SQL Analyzer}

The SQL Analyzer allows the user to execute SQL queries and view their results. The results of the query (if any) are displayed in a table in the user interface. The table's heading is updated to match the field names of the results.

The result table is a simple JTable. The DefaultTableModel object associated with the JTable allows for all manipulations of the table's data.

Each executed query returns a ResultSet object. The ResultSetMetaData object associated with the ResultSet instance provides the names for each of the returned columns. These column names are used for the table model's header. The table model is then populated one row at a time using the ResultSet.

\subparagraph{HQL Analyzer}

The HQL analyzer works just like the SQL analyzer in that the user can manually type in any query, click the execute query button, and view the results. Where as the SQL analyzer would display all of the results in a table, the results from the HQL queries are displayed in an object tree. The tree displays each object's hierarchy within a tree structure.

The tree is implemented as a customized JTree. Each of the objects returned by the Hibernate query are stored as the user object of a DefaultMutableTreeNode. Each of the properties of the object for which we have a getter and/or setter are further represented as children of the nodes. This recursive deepening goes until we reach the core Java classes.

\paragraph{End-Users}
Using the query analyzer is identical whether working with SQL or HQL. The query is entered in the text box on the north side of user interface. The radio buttons on the right allow the user to designate the entered query as either SQL or HQL. The results of an SQL query are displayed in a table format. Since and HQL query returns a set of objects, the results of an HQL query are displayed in an object tree in order to display the breakdown of each object.

\paragraph{Future Work}
We will now discuss some changes that are being considered for the next iteration of the utilities. The user interface and the object tree still have plenty of modifications that can be performed and these will be addressed in the near future.

1. Currently, the object tree is populated with all of the returned data upon executing the query. While this works fine for small sets of objects, this is not efficient for large data sets. For the next iteration we will maintain a copy of the returned objects locally, but we will only populate the first level of the tree. Each sub tree will be further populated as the user expands the corresponding tree nodes.

2. The user interface currently displays both the result table and the object tree simultaneously. The next update should display only one of these components at a time. Upon selecting each of the radio buttons designating which type of query we are working with, the proper component should be made visible. Selecting the SQL radio button shall display the result table and selecting the HQL radio button shall display the object tree.

3. The result table associated with SQL query results currently only displays
the data as it is returned. So modifications, manipulations, or sorting is
currently allowed. For the next version, we will utilize the the customized
JTable available in Dr. Dionisio's Shag package.
