\section{Background and Motivation}
With the introduction of DNA micro arrays.  The need for tools to aid in the processing of large amounts of Bioinformatic data are needed.  Data can now be collected in larger sizes than ever before and would be completely impractical if not impossible to analyze by hand.  GenMapp is a widely used tool in bioinformatics in order to visualize the changes in known pathways using data collected from DNA micro arrays. GenMapp is available to download for free. However GenMapp has one severe limitation.  It can only be used on organizms for which a GenMapp gene database already exists.  The current process for GenMapp gene database creation is to use Ensembl (www.ensembl.org) as a base database.   The problem with this approach is that the Ensembl database is very limited, covering only mammalian organisms.  By creating a process by which GenMapp databases for any organism may be created by anybody, without the need for indepth technical knowledge of databases, XML, GenMapp, etc., we have pushed the door to world of micro array analysis wide open.

\cite{genmapp:ng} \cite{mappfinder:gb} \cite{genmapp:bax}