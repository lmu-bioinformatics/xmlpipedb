\section{Background and Motivation}
With the introduction of DNA micro arrays.  the need for tools to aid in the processing of large amounts of Bioinformatic data are needed.  Data can now be collected in larger sizes than ever before and would be nearly impossible to analyze by hand.  GenMapp is a widely used tool in bioinformatics in order to visualize the changes in known pathways using data collected from DNA micro arrays.  However GenMapp has one severe  limitation.  It can only be used on organizms for which a GenMapp gene database already exist.  The current process for GenMapp gene database creation is to use ensamble as a base database.   The problem with this approach is that ensemble database is very limited and only covers mammalian organisms.  By creating a process by which 
\cite{genmapp:ng} \cite{mappfinder:gb} \cite{genmapp:bax}