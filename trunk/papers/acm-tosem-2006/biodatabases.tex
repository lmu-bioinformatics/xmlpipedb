\section{Application to Biological Databases}
\label{biodb}
% First three paragraphs added by Joe Boyle 4/30/2006
% Need Scott or Roberto to add information about GO
The GenMAPP application uses a gene database that contains species-specific 
libraries of gene information.  The database is needed in order to link expression
data with MAPPs, for creating and modifying MAPPs, and for importing new data.  
The gene database contains two main types of tables, Gene Tables and Relationship 
Tables.  A Gene table is a collection of gene identifiers obtained from a 
cataloging system for a particular species.  A Relationship Table provides a link 
between gene IDs of two separate gene ID databases.  These tables are essential to
the GenMAPP application~\cite{noSupport}.  

GenMAPP gene databases function around a central system, known as the MOD system.  
A Model Organism Database (MOD) is a public database that contains genes and 
annotation for a particular organism.  
The purpose of using a MOD system within GenMAPP is to form a link between
GenMAPP and the gene ontology (GO) hierarchy~\cite{noSupport}.  Current gene 
databases supplied by GenMAPP use the Ensembl database to link to GO.  However, 
Ensembl is limited to the number of species it represents, which is only 
mammalian organisms.  
As a result, for the XMLPipeDB project, UniProt was chosen to form the link to GO
based on the number of species available from it.  In this case, UniProt is not 
actually a MOD in the strict sense of the word.  But it can be used to form the 
needed link to the GO hierarchy.  

UniProt (Universal Protein Resource) is the world's most comprehensive catalog of 
information on proteins~\cite{uniprotWeb}.  
The European Bioinformatics (EBI) group and the
Swiss Institute of Bioinformatics (SIB) produced Swiss-Prot and TrEMBL 
(Translated EMBL Nucleotide Sequence Data Library)
whereas 
Protein Information Resource (PIR) produced the Protein Sequence Database (PIR-PSD).  
By noticing that the information contained in these separate databases was beginning
to overlap, the UniProt Consortium was formed.  
UniProt has now become the central repository 
for protein sequence and function after joining the information found in 
Swiss-Prot, TrEMBL, and PIR databases.  

