\subsection{Client Configuration}
In order to make the tools as easy to use as possible the issue of Hibernate properties configuration had to be addressed. Hibernate supports a wide variety of databases and data access methods. While advantageous for technical people, this can make configuration a daunting task for others. We choose an approach that would be modular, allowing the configuration component to be plugged into any application seamlessly. This was achieved by breaking the component in to four distinct objects: HibernateProperty, HibernatePropertiesModel, ConfigurationController and ConfigurationPanel. The ConfigurationController loads the structure of the Hibernate properties along with default values into a HibernatePropertiesModel. Each property is represented by a HibernateProperty object in the model. HibernateProperty objects consist of a category, type, name and value, as well as an indicator of whether the property is saved or not. This allows both saved and unsaved properties to be stored in the same model. The model is used by the ConfigurationPanel to render the a GUI for the user to view and/or edit the properties. When properties are saved, a model object is passed back to the controller, which then stores the properties in a hibernate.properties file.

