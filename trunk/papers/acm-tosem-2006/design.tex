\section{Overall Design and Approach}
To understand the overall design of XMLPipeDB we must first state our initial 
goal:  To create a reusable tool set that given genomic sequencing data for 
an organism in XML and a schema for that XML document could out put a working 
GenMAPP gene database for that organism.  

Given the many different source organizations, and consequently different schemas,  for this XML data we wanted to be able to automate this process as much as possible.  In the initial research phase of the project we managed to find that quite a fair amount of work had already been done by the open source community to reach our final goal.  The JAXB reference implementation contained an XML schema compiler to automatically create Java objects given an XML schema. Hyberjaxb had been built using JAXB and extending its functionality to Hibernate. Thus using Hyperjaxb, we could annotate these Java objects and automatically generate Hibernate mappings for them. This allowed us to use Hibernate almost "free of charge" since the work, which would have been considerable, to annotate the Java objects and generate the mapping files, was being done for us. We used hibernate to perform the object to relational mapping. Finally we could export the needed fields from our relational database to create GenMAPP data files. 

Initially the project seemed simple and we attempted to design it as a single unit. After some experimentation, we noted that some additional post processing was necessary to massage the Java, SQL and Hibernate mapping files for Uniprot and the SQL and Hibernate mapping files for GO. Given this, it made more sense to break the project up into three different stages.  The first stage, called xsd2db, described more thoroughly in \ref{xsd2db}, is a general purpose JAXB object and hibernate mapping generator that would take an XML schema file (XSD or DTD) as input.   

% Finish up the above section
% Added uniprotdb - Joe Boyle 4/30/2006

In the second stage, post-processor applications, UniprotDb and GoDb were written for the Uniprot and GO databases. This takes the output from Xsd2Db, makes the kinds of changes described above and then, through an ANT build file, allows the user to Jar the output for use a library with GenMAPP Builder.

The final stage, called GenMAPP Builder, uses the output of UniprotDb and GoDb and another module, XmlPipeDb Utilities, to provide the end user with an application via which s/he can: 
\begin{itemize}
	\item {configure the Hibernate properties for the intermediate database being used (for us it was PostgreSQL)}
	\item {import a Uniprot and/or GO XML file to the intermediate database; run queries on that data}
	\item {export the data to a GenMAPP 2 data file (MS Access MDB file with the GenMAPP 2 schema)}


The approach we propose to automatically migrate existing databases to a relational database and then use genmap builder to export a GenMAPP gene database allows us to use virtually any existing bioinformatics database that provides exports of their database in XML.  This allows us to expand GenMAPP to analyze plant and bacteria organisms.  

XmlPipeDb Utilities consists of components to:
 \begin{itemize}
	\item {Configure the database to import data and run queries via the configuration interface}
	\item {Import XML data into a database using the import interface}
	\item {Query the and view data with the query interface}
\end{itemize}

The advantage of this overall design is that each component or module can be used individually, with a completely different application or, as we have done here, as part of a chain of applications. Some of the components, like UniprotDb and GenMAPP Builder are very specific in the problem domain they address. Others, however, like Xsd2Db nad XmlPipeDb Utilities are very general and can be applied to any set of XSD and XML files the user wishes to use in conjuction with a relational database.
