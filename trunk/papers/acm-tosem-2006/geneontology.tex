\documentclass[12pt]{article}
\begin{document}


\subsection{Gene Ontology}
% Updated by Roberto and Scott
The Gene Ontology (henceforth Go) database is a staging database created to hold the records describing gene products according to biological processes,
cellular component, and molecular functions. This database allows the user to extract structured-controlled vocabulary terms
which describe gene products in any organism. In particular, Escherichia coli related data is queried as a part of the main
objective of the xmlpipedb project. Go exports its data to a number of different file formats to allow flexibility for different users with
different background and purposes. For this project, choosing a file format proved to be a difficult task.

\subsubsection{OBO.DTD Schema File}
\label{godtd}
There are three main file formats to extract Go data: flat files, mySQL formats, and XML files. For this project,
flat files were eliminate because they are not compatible with JaxB classes (Java Architecture for XML Binding), which  are called by
xmlpipedb utilities. MySQL format were eliminated because a "one-off" tool would need
to be created to port the data from mySQL to postgres. Initially, this format was considered since xsd2db
(see section \ref{xsd2db}) required a XSD
schema file (hence its name), which GO did not provide. A later version of xsd2db supported DTD schema files, which Go uses to define their
schema definitions. If we used mySQL, then what happens if the schema changes? The "one-off" tool would need to be changed every
time Go changed its schema, which would be costly and inefficient. But xsd2db provides the ability to auto generate any time Go changes
their schema, so the mySQL format was passed, which left XML formats.

Go provides their data in two XML format which both contain DTD schema files: OBO.xml and RDF.xml. However,
the members of the GO Consortium claimed that they will eventually generate a XSD schema for OBO.xml, and that there
are no plans to generate an XSD schema for the RDF file. In fact, they only reason why they currently have a DTD file
for RDF.xml is for historical reasons. Thus, the OBO.xml file format
was chosen for the project, and therefore the OBO.dtd file was fed into xsd2db.

As it turned out, the OBO.dtd file produced a SQL schema that contained a tabled name \emph{To}, which happens to be
a postgresSQL keyword. Furthermore, since the DTD file did not specify a maximum character length for all Go tags-value
pairs, the SQL schema defaulted those tags to type \emph{varchar(255)}. Unfortunately, though, some of the data exceeds the
character limit of 255. Thus, a godb postprocessor was created.

\subsubsection{Postprocessor}
The godb postprocessor is a GUI based tool to fix the errors in section~\ref{godtd}. Two files needed to be modified  before
godb can be delivered: \emph{schema.sql} and \emph{To.hbm.xml}. For the \emph{schema.sql} file, the postprocessor
searched and replaced all instances of \emph{varchar(255)} to \emph{varchar}, which specifies an unlimited character string type.
Thus, any data that exceeded 255 characters would not generate an error. Additionally,
the postprocessor changed the table name \emph{To} to  \emph{To\_}, which eliminates any conflicts with postgresSQL keywords.
Unfortunately, xsd2db
creates a xml hibernate mapping file that expects there to be a table named \emph{To}, which moves us on to the second file to modify,
\emph{To.hbm.xml}. Within this XML file, there is an \emph{table} element that defines the table name for this mapping, which had the
value of \emph{To}. Thus, the postprocessor simply replaced this value from \emph{To} to \emph{To\_}.

In the interest for flexibility, the postprocessor GUI asks the user to supply the file names to be modified.
Once each file has been selected, the tools proceeds to fix the corresponding files and returns a success message
if the changes were done correctly, otherwise the application terminates without modifying the files.

\subsubsection{Performance Analysis}.
Once the Godb ran through the postprocessor, a OBO.xml was loaded using the import features of xmlpipedb utilities.
Unfortunately, the import failed. Fortunately it was simply a memory heap issue: since the OBO.xml file contains alot of data, and
JaxB instantiates the "entire" object graph in memory before committing it to the database, an \emph{out of memory} error was issue.
To fix this, the java \emph{maximum heap size} was increased to 1024mb, and the subsequent import was successful.

The Godb served its purpose because all the data from gene ontology was successfully uploaded.  The database access was also
successfully executed by gmbuilder without any major problems.  The development of this part of the project was facilitated
in part because the uniprot database.  The functionality of the postprocessor was mirrored from the same postprocessor
developed by uniprot db.  By using the same code design as the uniprot postprocessor, the future maintenance of the Godb
code can be done in parallel with any changes done on the uniprot db side.

\subsubsection{Future Work}
The future work that can be accomplished with the Godb can be extensive.  However, in order to have a good understanding on
what needs to be done, developers must work together with biologists to find what is the relevant information they need to
access and how they can associate this information with other data. One issue relevant to the Godb postprocessor tool is
that is limited to the postgres database environment.  Further additions to the tool can be done to make the changes
according to what database is being use.  Source code modification and system Integration test would have to be done in
databases such as Oracle, SQL server, MySQl, and any other commonly used databases.  Another future work that can be done is
documentation.  Although the Godb project successfully performed its work, it was not documented properly.  A software
requirement specification can be prepared which lists all the requirements on the database and the postprocessor tool.  The
delivery of this document can give an idea to users and developers on how to proceed when using or modifying the Godb
applications.

\end{document}
